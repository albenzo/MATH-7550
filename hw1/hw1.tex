\documentclass[10pt]{article}
\usepackage[utf8]{inputenc}
\usepackage{amscd}
\usepackage{amsmath}
\usepackage{amssymb}
\usepackage{amsthm}
\usepackage{listings}
\usepackage{enumerate}

\textwidth=15cm \textheight=22cm \topmargin=0.5cm \oddsidemargin=0.5cm \evensidemargin=0.5cm

\newcommand{\sk}{\vskip 10mm}
\newcommand{\bb}[1]{\mathbb{#1}}
\newcommand{\ra}{\rightarrow}

\theoremstyle{plain}
\newtheorem{problem}{Problem}
\newtheorem{lemma}{Lemma}[problem]

\theoremstyle{remark}
\newtheorem{tpart}{}[problem]
\newtheorem*{ppart}{}

\begin{document}

\begin{problem}[1]
  Show that a second countable Hausdorff space $X$ with a functional
  structure $F$ is an $n$-manifold if, and only if, every point in
  $X$ has a neighborhood $U$ such that there are functions
  $f_1,\ldots,f_n\in F(U)$ such that: a real valued function on $g$
  on $U$ is in $F(U)$ if, and only if, there exists a smooth function
  $h(x_1,\ldots,x_n)$ of $n$ real variables $\ni g(p)=h(f_1(p),\ldots,f_n(p))$
  for all $p\in U$.
\end{problem}

\begin{proof}
  Let $X$ be a second countable Hausdorff space with a functional
  structure $F$. Suppose that $X$ is an $n$-manifold and let
  $x\in X$ be a point with $U\subset X$ an open neighborhood of $X$.
  Then by definition we have an algebra $F(U)$. Since $F(U)$
  is an algebra we have a basis $(f_1,\ldots,f_n)$. Let
  $g(p)=\sum c_if_i(p)$ for $p\in U$. Define $h=(x_1,\ldots,x_n)=\sum c_ix_i$.
  Then $g(p)=h(f_1(p),\ldots,f_n(p))$ for all $p\in U$. If $g$ were
  not in $F(U)$ then it would not be smooth. As such it could
  not be of the form $h(f_1(p),\ldots,f_n(p))$ as if it were it
  would be smooth. Therefore if $X$ is an $n$-manifold
  then $g\in F(U)$ if, and only if, there is a smooth $h$ of $n$ real
  variables such that $g(p)=h(f_1(p),\ldots,f_n(p))$. 

  Now suppose that given a real valued $g$ on $U$ that $g\in F(U)$
  if, and only if, there is a smooth function $h$ such that
  $g(p)=h(f_1(p),\ldots,f_n(p))$ for all $p\in U$. Let $x\in X$ and
  let $U$ be the corresponding neighborhood of $x$ where the
  above holds. Define $\phi:X\rightarrow V\subset\bb{R}^n$ via
  \[\phi(p)=(f_1(p),\ldots,f_n(p)) \]
  This map is continuous since each $f_i$ is continuous. Moreover
  if $h\in C^\infty(V)$ then
  \[ h\circ \phi=h(f_1,\ldots,f_n)\]
  which places $h\circ \phi\in F_X(U=\phi^{-1}(V))$ by our assumption. Thus
  $\phi$ is a morphism of functional structures from $(X,F_X(U))$
  to $(V,C^\infty)$.

  I tried to use Hausdorffness to force it to be injective as
  well as trying to coerce the inverse function theorem to work
  in this scenario. However I was unsucessful, but I'll continue
  on without.
  
  Thus $\phi^{-1}$ has a well defined inverse.

  Then
  \[ g\circ\phi^{-1}(x)=h(f_1\circ\phi^{-1}(x),\ldots,f_n\circ \phi^{-1}(x)) \in F(V) \]
  Which shows that $\phi$ is in fact an isomorphism on functionally
  structured spaces and as such $X$ is locally isomorphic to
  $(\bb{R}^n,C^\infty)$ and thus a smooth manifold.
\end{proof}

\sk

\begin{problem}[3]
  Show that a map $f:M\rightarrow N$ between smooth manifolds, with functional
  structures $F_M$ and $F_N$, is smooth in the sense of definition 2.5
  if, and only if, it is smooth in the sense of definition 2.4.
\end{problem}

\begin{proof}
  First note that given a chart $\phi:U\rightarrow\bb{R}^n$ that $\phi$ is a diffeomorphism
  from $U$ to $\phi(U)$ with the smooth structure on $\phi(U)$ being generated
  by the identity map $id_U$. To verify this we have that 
  $id_U\circ\phi\circ\phi^{-1}=id_U$ is smooth since the $id_U$ is smooth. Checking that $\phi^{-1}$
  is smooth is similar as we check $\phi\circ\phi^{-1}\circ id_U=id_U$ which is smooth.
  
  Let $\phi,\psi$ be charts for $M,N$ respectively and let $f:M\rightarrow N$ be a map that
  is smooth in the sense that if $g\in F_N(V)$ for $V\subset N$ open then
  $g\circ f\in F_M(f^{-1}(V))$. Consider $\psi\circ f\circ \phi^{-1}$. Since both $\phi,\psi$ are diffeomorphisms
  from their domain to their image they and their inverses will be smooth. It then
  follows that if $V$ is the domain of $\psi$ that $\psi\in F_N(V)$. By
  our assumption we then have that $\psi\circ f\in F_M(f^{-1}(V))$ implying that
  $\psi\circ f$ is smooth. Since smoothness of functions is preserved by composition
  we have that $\psi\circ f\circ\phi^{-1}$ is smooth.

  Now suppose that $f:M\rightarrow N$ was smooth in the sense that $\psi\circ f\circ \phi^{-1}$ is smooth
  wherever it is defined. Due to the union property of functional structures
  it will suffice to show that $g\in F_N(V)$ implies $g\circ f\in F_M(f^{-1}(V))$
  for a $V$ sufficiently small such that both $V$ and $f^{-1}$ fall in the
  domains of single charts $\psi,\phi$. Let $g\in F_N(V)$. Then $g$ is smooth
  which implies that $g\circ\psi^{-1}$ is smooth. By our assumption
  $\psi\circ f\circ \phi^{-1}$ is smooth. Then if we compose we get that
  \[ (g\circ\psi^{-1})\circ(\psi\circ f\circ\phi^{-1})\circ\phi =g\circ f\]
  is smooth. Thus $g\circ f\in F_M(f^{-1}(V))$.

  Therefore the two definitions for smooth maps are equivalent.
\end{proof}

\sk

\begin{problem}[4]
  Let $X$ be the graph of a real valued function $\theta(x)=|x|$ of a real
  variable $x$. Define a functional structure on $X$ by taking $f\in F(U)$
  if, and only if, $f$ is the restriction to $U$ of a $C^\infty$ function on some
  open set $V$ in the plane with $U=V\cap X$. Show that $X$ with this structure
  is \textit{not} diffeomorphic to the real line with the usual $C^\infty$ structure.
\end{problem}

Let $f(x,y):\bb{R}^2\rightarrow \bb{R}$ be $x+y$ rotated counterclockwise $\pi/4$ so that the former $x,y$ axes
line up with $U$. Then $f|_U\equiv |r|$ for $r\in U$ which implies that $|\cdot|\in F(X)$ however
this function is not smooth on the real line. Thus $X$ and $\bb{R}$ with the usual
structure are not diffeomorphic.


%%%%%%%%%%%%%%%%%%%%%%%%%%%%%%%%%%%%%%%%%%%%%%%%%%%%%%%%%%%%%%%%%%%%%%%%%%%%%
\end{document}

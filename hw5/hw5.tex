\documentclass[10pt]{article}
\usepackage[utf8]{inputenc}
\usepackage{amscd}
\usepackage{amsmath}
\usepackage{amssymb}
\usepackage{amsthm}
\usepackage{listings}
\usepackage{enumerate}

\textwidth=15cm \textheight=22cm \topmargin=0.5cm \oddsidemargin=0.5cm \evensidemargin=0.5cm

\newcommand{\sk}{\vskip10mm}
\newcommand{\bb}[1]{\mathbb{#1}}
\newcommand{\ra}{\rightarrow}

\theoremstyle{plain}
\newtheorem{problem}{Problem}
\newtheorem{lemma}{Lemma}[problem]

\theoremstyle{remark}
\newtheorem{tpart}{}[problem]
\newtheorem*{ppart}{}

\begin{document}

\begin{problem}
  Consider the real valued function $f(x,y,z)=(2-(x^2+y^2)^{1/2})^2+z^2$
  on $\bb{R}^3\setminus\{(0,0,z)\}$. Show that $1$ is a regular value of $f$.
  Identify the manifold $M=f^{-1}(1)$.
\end{problem}

First we calculate the partials
\[
  \frac{\partial f}{\partial x}=x\left(2-\frac{4}{(x^2+y^2)^{1/2}}\right) ,
  \quad\frac{\partial f}{\partial y}=y\left(2-\frac{4}{(x^2+y^2)^{1/2}}\right) ,
  \quad\frac{\partial f}{\partial z}= 2z 
\]

The critical points of $f$ are those for which the rank of $f_*$ is zero. This
requires that $z=0$. So if $1$ were to be a critical value it would have
to occur when $(x^2+y^2)^{1/2}$ is either $1$ or $3$. However neither of these would
force all the partial derivatives to be zero. As such $1$ is a regular value of $f$.

The manifold is the torus $T^2$.

\sk

\begin{problem}
  Show that the manifold $M$ of Problem 1 is transverse to the surface
  \[
    N=\{(x,y,z)\in\bb{R}^3|x^2+y^2=4\}
  \]
  Identify the manifold $M\cap N$.
\end{problem}

The points of intersection are
\[
  N\cap M = \{(x,y,z)|z=\pm 1, x^2+y^2=4\}
\]

Take the normal vector to $M$ at one of the points of intersection using
the partial derivatives above to get $\langle 0,0,\pm 2\rangle$ where the positive is
the top circle and the bottom circle is the negative. Similarly the normal
vector to $N$ is $\langle 2x,2y,0\rangle$. Since both $x$ and $y$ cannot be zero on the
intersection and the normal vectors for each surface are not parallel
we have that for $p\in M\cap N$ that $T_p M+T_p N=T_p\bb{R}^3$ which shows
that $M$ and $N$ are transverse.


The manifold is two disjoint copies of $S^1$.
\sk

\begin{problem}
  Show that the manifold $M$ of Problem 1 is not transverse to the surface
  \[
    N=\{(x,y,z)\in\bb{R}^3|x^2+y^2=1\}
  \]
  Is $M\cap N$ a manifold?
\end{problem}

The points of intersection are
\[
  N\cap M = \{(x,y,z)=z=0,x^2+y^2=1\}
\]

Using the partials we calculated in problem one. The normal vector for
any point $(x,y,0)\in T^2$ along the intersection will be $\langle-2x,-2y,0 \rangle$.
Similarly for $N$ the normal vector will be $\langle 2x,2y,0\rangle$. However the
planes defined by these normal vectors will be parallel and as such their
sum cannot be $\bb{R}^3$. Thus the manifolds $M$ and $N$ are not transverse.

It is a manifold and it is a single copy of $S^1$.

%%%%%%%%%%%%%%%%%%%%%%%%%%%%%%%%%%%%%%%%%%%%%%%%%%%%%%%%%%%%%%%%%%%%%%%%%%%%%
\end{document}

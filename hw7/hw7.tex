\documentclass[10pt]{article}
\usepackage[utf8]{inputenc}
\usepackage{amscd}
\usepackage{amsmath}
\usepackage{amssymb}
\usepackage{amsthm}
\usepackage{listings}
\usepackage{enumerate}
\usepackage[all,cmtip]{xy}

\textwidth=15cm \textheight=22cm \topmargin=0.5cm \oddsidemargin=0.5cm \evensidemargin=0.5cm

\newcommand{\sk}{\vskip 10mm}
\newcommand{\bb}[1]{\mathbb{#1}}
\newcommand{\ra}{\rightarrow}

\theoremstyle{plain}
\newtheorem{problem}{Problem}
\newtheorem{lemma}{Lemma}[problem]

\theoremstyle{remark}
\newtheorem{tpart}{}[problem]
\newtheorem*{ppart}{}

\begin{document}

\begin{problem}
  On an open set $U\subset \bb{R^n}$ show that the exterior derivative $d$ is
  the only operator $d:\Omega^p(U)\rightarrow\Omega^{p+1}(U)$ satisfying:
  \begin{enumerate}
  \item[(a)] $d(\omega+\eta)=d\omega+d\eta$
  \item[(b)] $\omega\in\Omega^p(U),\eta\in\Omega^q(U)\Rightarrow
    d(\omega\wedge \eta)=d\omega\wedge\eta+(-1)^p\omega\wedge d\eta$
  \item[(c)] $f\in\Omega^0(U)\Rightarrow df(X)=X(f)$
  \item[(d)] $f\in\Omega^0(U)\Rightarrow d(df)=0$
  \end{enumerate}
  Deduce that $d$ is independent of the coordinate system used to define it.
\end{problem}

\begin{proof}
  First note that due to the linearity of $d$ (pull out a bump function)
  that given a point $p$ the value of $d\omega_p$ only relies on a
  neighborhood of $p$.

  Now suppose that $d'$ was another operator that fulfilled the same
  properties as $d$. Then given $\omega=fdx_{i_1}\wedge\cdots\wedge dx_{i_p}\in\Omega^p(U)$
  through repeated applications of (b) we have
  \[
    d'(fdx_{i_1}\wedge\cdots\wedge dx_{i_p})=d'fdx_{i_1}\wedge\cdots\wedge dx_{i_p} -\sum_1^p(-1)^pfdx_{i_1}\wedge\cdots\wedge d'dx_{i_j}\wedge\cdots\wedge dx_{i_p}
  \]
  However $d'dx_{i_j} = d'd'x_{i_j}= 0$ and $d'f=df$ hold by (c), (d) and the fact that
  $d$ applied to a zero-form is not affected by coordinates we have that
  \[
    d'fdx_{i_1}\wedge\cdots\wedge dx_{i_p} -\sum_1^p(-1)^pfdx_{i_1}\wedge\cdots\wedge d'dx_{i_j}\wedge\cdots\wedge dx_{i_p} = df\wedge dx_{i_1}\wedge\cdots\wedge dx_{i_p}=d(f dx_{i_1}\wedge\cdots\wedge dx_{i_p})
  \]
  which shows that $d = d'$.

  Since $d$ is unique and at a point only depends on a neighborhood of said point
  given two neighborhoods of a point $p$ called $U,V$ even if they have different
  coordinate systems must agree on $U\cap V$. Thus it must be the case that $d$ is
  independent of coordinate systems.
\end{proof}

\sk

\begin{problem}
  On the unit circle $S^1$ in the plane, let $\theta=\arctan(y/x)$ be the usual
  polar coordinate. Show that $d\theta$ makes sense on $S^1$ and is a closed
  1-form which is not exact.
\end{problem}

If we calculate $d\theta$ we get
\[
  d\theta = \frac{\partial \theta}{\partial x}dx+\frac{\partial\theta}{\partial y}dy = \frac{-ydx}{x^2+y^2} + \frac{xdy}{x^2+y^2}
\]

However since we are on the unit circle the bottom term becomes 1. Thus
$d\theta=-ydx+xdy$. Since $\theta$ is not continuous when $x=0$ we cannot directly
apply $d^2=0$ to $\theta$. However there are only two points where $\theta$ is
discontinuous. If we throw out those points we have $d^2\theta=0$. Since
$d^2\theta$ is a smooth 2-form on $S^1$ we cannot have nonzero points when
$x=0$ as that would violate the smoothness. Thus $d^2\theta$ is uniformly
zero which shows that $d\theta$ is closed.

To see that it is not closed evaluate the integral $\int_{S^1}d\theta$. If it
were closed the integral would be zero however it is $2\pi$. Therefore
$\theta$ is not exact.

\pagebreak

\begin{problem}
  For $\omega\in\Omega^1(M)$, verify the special case
  $d\omega(X,Y)=X(\omega(Y))-Y(\omega(X))-\omega([X,Y])$ of the invariant
  formula mentioned above Definition 2.4.
\end{problem}

\begin{proof}
  Let $\omega=fdx\in\Omega^1(M)$. Then
  \begin{align*}
    d\omega(X,Y) &= d(fdx)(X,Y) \\
            &= df\wedge dx(X,Y) \\
            &= df(X)dx(Y)-df(Y)dx(X) \\
            &= XfYx-YfXx \\
            &= (XfYx+fXYx)-(YfXx+fYXx)-f(XYx-YXx) \\
            &= X(fYx)-Y(fXx)-f[X,Y]x \\
            &= X(\omega(Y))-Y(\omega(X))-\omega([X,Y])\\
  \end{align*}
  which completes the proof.
\end{proof}

%%%%%%%%%%%%%%%%%%%%%%%%%%%%%%%%%%%%%%%%%%%%%%%%%%%%%%%%%%%%%%%%%%%%%%%%%%%%%
\end{document}

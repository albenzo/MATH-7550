\documentclass[10pt]{article}
\usepackage[utf8]{inputenc}
\usepackage{amscd}
\usepackage{amsmath}
\usepackage{amssymb}
\usepackage{amsthm}
\usepackage{listings}
\usepackage{enumerate}

\textwidth=15cm \textheight=22cm \topmargin=0.5cm \oddsidemargin=0.5cm \evensidemargin=0.5cm

\newcommand{\sk}{\vskip 10mm}
\newcommand{\bb}[1]{\mathbb{#1}}
\newcommand{\ra}{\rightarrow}

\theoremstyle{plain}
\newtheorem{problem}{Problem}
\newtheorem{lemma}{Lemma}[problem]

\theoremstyle{remark}
\newtheorem{tpart}{}[problem]
\newtheorem*{ppart}{}

\begin{document}

\begin{problem}[1]
  For the map $\phi(x)=x\sin(x)$ of the real line to itself,
  what are the regular values?
\end{problem}

Since $\phi$ is a map from $\bb{R}$ to $\bb{R}$ we can acquire
the critical points via the usual method from calculus.
Thus the regular values the points that are not solutions to
\[
  \sin(x)+x\cos(x) =0
\]

\sk

\begin{problem}[2]
  For the map $\phi(x,y)=x^2-y^2$ of the plane to the line, what
  are the regular values?
\end{problem}

The pushforward for $\phi$ will be $\phi_*(x,y)=(2x,-2y)$. Since we are
mapping from 2 dimensions to 1 the regular values will
be those where $\phi_*$ is surjective. This will be the points
where neither $x$ nor $y$ are 0.

\sk

\begin{problem}[3]
  For the map $\phi(x,y)=\sin(x^2+y^2)$ of the plane to the line,
  what are the regular values?
\end{problem}

First rewrite the function as $\phi(r)=\sin(r^2)$. Then if
we take the derivative we get $\phi'(r)=2r\cos(r^2)$. Thus
the critical values are when $r=0$ or when $r=\pm\sqrt{\frac{(2k+1)\pi}{2}}$
for $k\in \bb{Z}$. Translating back to $x,y$ we have that the
regular values are when the equations $x^2+y^2=0$ and
$x^2+y^2=\frac{(2k+1)\pi}{2}$ for $k\in \bb{Z}$ do not hold.

\sk

\begin{problem}[5]
  Let $\gamma:\bb{R}\rightarrow\bb{R}^2$ be a smooth curve in the plane.
  Let $K$ be the set of all $r\in \bb{R}$ such that the circle of
  radius $r$ about the origin is tangent to the curve $\gamma$ at
  some point. Show that $K$ has empty interior in $\bb{R}$.
\end{problem}

\begin{proof}
  Define $\delta(t)=|\gamma(t)|$. The function $\delta$ is smooth so long as
  $\gamma$ avoids the origin. However since $\gamma$ is smooth this can
  only happen either a finite number of times or if $\gamma$ constant
  at the origin. In the latter case $K=\emptyset$.

  In the other case however $\bb{R}\setminus\{t\in \bb{R}|\gamma(t)\}$ is open
  and as such a manifold. Thus we have that $\delta$ is smooth and
  we can apply Sard's theorem. Then the critical points of
  $\delta$ are precisely the points in $K$ without the removed points
  above. By Sard's theorem
  the critical points of $\delta$ are of measure zero. Thus
  $K$ is the union of a finite set and a measure zero set and as
  such is of measure zero and cannot contain any intervals. It
  then follows that no point in $K$ has an open neighborhood
  around it and as such the interior of $K$ is empty.
\end{proof}

\sk

\begin{problem}[6]
  If $C$ is a circle embedded smoothly in $\bb{R}^4$, show that
  there exists a three-dimensional hyperplane $H$ such that the
  orthogonal projection of $C$ to $H$ is an embedding.
\end{problem}

\begin{proof}
  Let $\gamma:S^1\rightarrow\bb{R}^4$ be a smooth embedding of $S^1$ into $\bb{R}^4$.
  There are two obstacles to finding a hyperplane for which the
  projection of $\gamma$ onto $H$ is an embedding. The first is that
  we need to preserve injectivity of the function itself and the
  second is that we need to ensure that it remains an immersion.

  To deal with the latter define a function
  \[
    f(t):S^1\rightarrow S^3
  \]
  via $f(t)=\frac{\gamma'(t)}{|\gamma'(t)|}$. This function will give us the
  unit tangent vector to $\gamma$ at time $t$. Moreover it is well defined
  and smooth since $\gamma$ is an embedding and as such there will be
  no point wherein $|\gamma'(t)|=0$.

  Next define $g(s,t):S^1\times S^1\setminus\{(t,t)|t\in S^1\}\rightarrow S^3$ as
  \[
    g(s,t) = \frac{\gamma(s)-\gamma(t)}{|\gamma(s)-\gamma(t)|}
  \]
  This function will also be well defined and smooth since $\gamma$ is an embedding
  and due to us explicitly removing points where $s=t$.

  Since translating a 3-dimensional hyperplane would not affect the projection
  of $\gamma$ we can specify each hyperplane $H$ by its unit normal vector $n$ which is
  an element of $S^3$.

  When projecting $\gamma$ onto a hyperplane with normal vector $n$ we break
  the immersion property if $f(t)=n$ for any $t$. Similarly we break
  injectivity if $g(s,t)=n$ for any $s,t$ in the domain of $g$. However
  since $S^3$ is a 3-manifold, $S^1$ is a 1-manifold, $S^1\times S^1\setminus\{(t,t)|t\in S^1\}$
  is a 2-manifold, and the fact that Sard's theorem implies that
  space filling curves are not smooth it must be the case that
  there is a vector $n\in S^3$ such that $n$ is not in the image of either $f$
  or $g$. It then follows that the projection of $\gamma$ onto the hyperplane
  specified by $n$ is an embedding.

  Therefore if $C$ is a smoothly embedded circle in $\bb{R}^4$ there
  exists a 3-dimensional hyperplane $H$ such that the projection of $C$
  to $H$ is an embedding.
\end{proof}

\sk

%%%%%%%%%%%%%%%%%%%%%%%%%%%%%%%%%%%%%%%%%%%%%%%%%%%%%%%%%%%%%%%%%%%%%%%%%%%%%
\end{document}

\documentclass[10pt]{article}
\usepackage[utf8]{inputenc}
\usepackage{amscd}
\usepackage{amsmath}
\usepackage{amssymb}
\usepackage{amsthm}
\usepackage{listings}
\usepackage{enumerate}
\usepackage[all,cmtip]{xy}

\textwidth=15cm \textheight=22cm \topmargin=0.5cm \oddsidemargin=0.5cm \evensidemargin=0.5cm

\newcommand{\sk}{\vskip 10mm}
\newcommand{\bb}[1]{\mathbb{#1}}
\newcommand{\ra}{\rightarrow}

\theoremstyle{plain}
\newtheorem{problem}{Problem}
\newtheorem{lemma}{Lemma}[problem]

\theoremstyle{remark}
\newtheorem{tpart}{}[problem]
\newtheorem*{ppart}{}

\begin{document}

\begin{problem}
  Deduce from de Rham's Theorem that if $U\subset\bb{R}^3$ is open and
  $H^1(U)=0$ then any vector field $\vec{F}$ on $U$ with
  $\text{curl\ }\vec{F}=0$ is a gradient field.
\end{problem}

\begin{proof}
  If there was a vector field with zero curl that was not a gradient
  field that would imply that the $H^1(U)\neq 0$. As such the above property
  must hold for any vector field.
\end{proof}

\sk

\begin{problem}
  Deduce from de Rham's Theorem that if $U\subset\bb{R}^3$ is open and
  $H^2(U)=0$ then any vector field $\vec{F}$ on $U$ with $\text{div\ }\vec{F}=0$
  has the form $\vec{F}=\text{curl\ }\vec{G}$ for some vector field
  $\vec{G}$ on $U$.
\end{problem}

\begin{proof}
  Similar to above if there were such a vector field that would imply that
  $H^2(U)\neq 0$. As such if a vector field has zero divergence then it
  must be the curl of some vector field.
\end{proof}

%%%%%%%%%%%%%%%%%%%%%%%%%%%%%%%%%%%%%%%%%%%%%%%%%%%%%%%%%%%%%%%%%%%%%%%%%%%%%
\end{document}

\documentclass[10pt]{article}
\usepackage[utf8]{inputenc}
\usepackage{amscd}
\usepackage{amsmath}
\usepackage{amssymb}
\usepackage{amsthm}
\usepackage{listings}
\usepackage{enumerate}

\textwidth=15cm \textheight=22cm \topmargin=0.5cm \oddsidemargin=0.5cm \evensidemargin=0.5cm

\newcommand{\sk}{\vskip 10mm}
\newcommand{\bb}[1]{\mathbb{#1}}
\newcommand{\ra}{\rightarrow}

\theoremstyle{plain}
\newtheorem{problem}{Problem}
\newtheorem{lemma}{Lemma}[problem]

\theoremstyle{remark}
\newtheorem{tpart}{}[problem]
\newtheorem*{ppart}{}

\begin{document}

\begin{problem}[2]
  If the curve $\phi:\bb{R}\rightarrow\bb{R}^n$ is an embedding then show that
  $\phi_*(d/dt)$ coincides with the classical notion of the tangent
  vector to the curve $\phi$ under the identification of the tangent
  space to a euclidean space with the euclidean space.
\end{problem}

\begin{proof}
  The traditional definition of a tangent vector in Euclidean space
  for $\phi(t)=(\phi_1(t),\ldots,\phi_n(t))$ would be
  \[
    \phi'(t)=(\phi_1'(t),\ldots,\phi_n'(t))
  \]
  However if we identify the tangent space as euclidean space
  then the pushforward $\phi_*:\bb{R}\rightarrow\bb{R}^n$ will be
  \[
    \phi_*\left(\frac{d}{dt}\right)=\frac{d\phi}{dt}=(\phi_1',\ldots,\phi_n')
  \]
  which shows that the two notions agree.
\end{proof}

\sk

\begin{problem}[3]
  For a smooth function $f$ defined on a neighborhood of a point
  $p\in \bb{R}^n$, the gradient $\nabla f=\mathrm{grad}f$ of $f$ is the vector
  \[
    \left\langle
      \frac{\partial f}{\partial x_1},\ldots,\frac{\partial f}{\partial x_n}
    \right\rangle
  \]
  For a vector $v\in\bb{R}^n$ show that the directional derivative $D_v$,
  denoted by $D_{\gamma_v}$ where $\gamma_v(t)=p+tv$, satisfies the equation
  \[
    D_vf =\langle \nabla f,v\rangle
  \]
  the standard inner product of $\nabla f$ with $v$ in $\bb{R}^n$.
\end{problem}

\begin{proof}
  First note that by definition that
  \[
    \langle \nabla f,v\rangle=\frac{\partial f}{\partial x_i}v^i
  \]
  Likewise for the other term we have
  \[
    D_{\gamma_v}f=\frac{d}{dt}f(\gamma_v(t))|_{t=0}=\frac{\partial f}{\partial x_i}\frac{d\gamma_v}{dt}|_{t=0}=\frac{\partial f}{x_i}v^i
  \]
  which completes the proof.
\end{proof}

\sk

\begin{problem}[4]
  If $M^m\subset\bb{R}^n$ is a smoothly embedded manifold and $f$
  is a smooth real valued function defined on a neighborhood of
  $p\in M^m$ in $\bb{R}^n$ and which is constant on $M$, show that $\nabla f$
  is perpendicular to $T_p(M)$ at $p$.
\end{problem}

\begin{proof}
  Since we are working in a neighborhood of $p\in M^m$ we can work in
  local coordinates. Then the gradient of $f$ will be
  $\nabla f= \langle\frac{\partial f}{x_1},\ldots,\frac{\partial f}{x_n}\rangle$. Moreover
  we have that $\nabla f(x)=0$ for $x\in M^m$ since $f$ is constant on $M^m$.

  Let $\sum a^i\frac{\partial}{\partial x_i}$ be in $T_p(M^m)$. Then we have that
  \[
    \nabla f(x)\cdot\sum a_i\frac{\partial}{x_i}=0
  \]
  which implies that $\nabla f$ is normal to $T_p(M^m)$.
\end{proof}

\sk

%%%%%%%%%%%%%%%%%%%%%%%%%%%%%%%%%%%%%%%%%%%%%%%%%%%%%%%%%%%%%%%%%%%%%%%%%%%%%
\end{document}

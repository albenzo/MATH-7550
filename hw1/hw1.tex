\documentclass[10pt]{article}
\usepackage[utf8]{inputenc}
\usepackage{amscd}
\usepackage{amsmath}
\usepackage{amssymb}
\usepackage{amsthm}
\usepackage{listings}
\usepackage{enumerate}

\textwidth=15cm \textheight=22cm \topmargin=0.5cm \oddsidemargin=0.5cm \evensidemargin=0.5cm

\newcommand{\sk}{\vskip 10mm}
\newcommand{\bb}[1]{\mathbb{#1}}
\newcommand{\ra}{\rightarrow}

\theoremstyle{plain}
\newtheorem{problem}{Problem}
\newtheorem{lemma}{Lemma}[problem]

\theoremstyle{remark}
\newtheorem{tpart}{}[problem]
\newtheorem*{ppart}{}

\begin{document}

\begin{problem}[1]
  Show that a second countable Hausdorff space $X$ with a functional
  structure $F$ is an $n$-manifold if, and only if, every point in
  $X$ has a neighborhood $U$ such that there are functions
  $f_1,\ldots,f_n\in F(U)$ such that: a real valued function on $g$
  on $U$ is in $F(U)$ if, and only if there exists a smooth function
  $h(x_1,\ldots,x_n)$ of $n$ real variables $\ni g(p)=h(f_1(p),\ldots,f_n(p))$
  for all $p\in U$.
\end{problem}

\begin{proof}
  
\end{proof}

\sk

\begin{problem}[3]
  Show that a map $f:M\rightarrow N$ between smooth manifolds, with functional
  structures $F_M$ and $F_N$, is smooth in the sense of definition 2.5
  if, and only if, it is smooth in the sense of definition 2.4.
\end{problem}

\begin{itemize}
\item[2.3:] A morphism of functionally structured spaces
  \[ (X,F_X)\rightarrow (Y,F_Y) \]
  is a map $\phi:X\rightarrow Y$ such that the composition $f\mapsto f\circ\phi$ carries
  $F_Y(U)$ into $F_X(\phi^{-1}(U))$.
\item[2.4:] A morphism is a smooth map if it is a morphism of functionally
  structured spaces for smooth manifolds as functionally structured spaces.
\item[2.5:] A map $f:M\rightarrow N$ between two smooth manifolds is said to be
  smooth if, for any charts $\phi$ on $M$ and $\psi$ on $N$, the function
  $\psi\circ f\circ\phi^{-1}$ is smooth where it is defined.
\end{itemize}

\begin{proof}
  Let $\phi,\psi$ be charts for $M,N$ respectively and let $f:M\rightarrow N$ be a map that
  is smooth in the sense that if $g\in F_N(V)$ for $V\subset N$ open then
  $g\circ f\in F_M(f^{-1}(V))$. Consider $\psi\circ f\circ \phi^{-1}$. Since both $\phi,\psi$ are diffeomorphisms
  from their domain to their image they and their inverses will be smooth. It then
  follows that if $V$ is the domain of $\psi$ that $\psi\in F_N(V)$. By
  our assumption we then have that $\psi\circ f\in F_M(f^{-1}(V))$ implying that
  $\psi\circ f$ is smooth. Since smoothness of functions is preserved by composition
  we have that $\psi\circ f\circ\phi^{-1}$ is smooth.

  Now suppose that $f:M\rightarrow N$ was smooth in the sense that $\psi\circ f\circ \phi^{-1}$ is smooth
  wherever it is defined. Let $V\subset N$ and let $V_\alpha:=V\cap U_\alpha$ where $\psi_\alpha:U_\alpha\rightarrow\bb{R}$
  is a chart of $N$. Then because of the union property of charts it will suffice
  to show that the definition of smoothness holds for one such $V_\alpha$. Let
  $g\in F_N(V_\alpha)$. Then $g\circ f\in F_M(f^{-1}(V_\alpha))$.
\end{proof}

\sk

\begin{problem}[4]
  Let $X$ be the graph of a real valued function $\theta(x)=|x|$ of a real
  variable $x$. Define a functional structure on $X$ by taking $f\in F(U)$
  if, and only if, $f$ is the restriction to $U$ of a $C^\infty$ function on some
  open set $V$ in the plane with $U=V\cap X$. Show that $X$ with this structure
  is \textit{not} diffeomorphic to the real line with the usual $C^\infty$ structure.
\end{problem}

\begin{proof}
  
\end{proof}

\sk

%%%%%%%%%%%%%%%%%%%%%%%%%%%%%%%%%%%%%%%%%%%%%%%%%%%%%%%%%%%%%%%%%%%%%%%%%%%%%
\end{document}

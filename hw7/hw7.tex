\documentclass[10pt]{article}
\usepackage[utf8]{inputenc}
\usepackage{amscd}
\usepackage{amsmath}
\usepackage{amssymb}
\usepackage{amsthm}
\usepackage{listings}
\usepackage{enumerate}
\usepackage[all,cmtip]{xy}

\textwidth=15cm \textheight=22cm \topmargin=0.5cm \oddsidemargin=0.5cm \evensidemargin=0.5cm

\newcommand{\sk}{\vskip 10mm}
\newcommand{\bb}[1]{\mathbb{#1}}
\newcommand{\ra}{\rightarrow}

\theoremstyle{plain}
\newtheorem{problem}{Problem}
\newtheorem{lemma}{Lemma}[problem]

\theoremstyle{remark}
\newtheorem{tpart}{}[problem]
\newtheorem*{ppart}{}

\begin{document}

\begin{problem}
  On an open set $U\subset \bb{R^n}$ show that the exterior derivative $d$ is
  the only operator $d:\Omega^p(U)\rightarrow\Omega^{p+1}(U)$ satisfying:
  \begin{enumerate}
  \item[(a)] $d(\omega+\eta)=d\omega+\eta$
  \item[(b)] $\omega\in\Omega^p(U),\eta\in\Omega^q(U)\Rightarrow
    d(\omega\wedge \eta)=d\omega\wedge\eta+(-1)^p\omega\wedge d\eta$
  \item[(c)] $f\in\Omega^0(U)\Rightarrow dfX)=X(f)$
  \item[(d)] $f\in\Omega^0(U)\Rightarrow d(df)=0$
  \end{enumerate}
  Deduce that $d$ is independent of the coordinate system used to define it.
\end{problem}

\begin{proof}
  
\end{proof}

\sk

\begin{problem}
  On the unit circle $S^1$ in the plane, let $\theta=\arctan(y/x)$ be the usual
  polar coordinate. Show that $d\theta$ makes sense on $S^1$ and is a closed
  1-form which is not exact.
\end{problem}

\begin{proof}
  
\end{proof}

\sk

\begin{problem}
  For $\omega\in\Omega^1(M)$, verify the special case
  $d\omega(X,Y)=X(\omega(Y))-Y(\omega(X))-\omega([X,Y])$ of the invariant
  formula mentioned above Definition 2.4.
\end{problem}

\begin{proof}
  
\end{proof}

%%%%%%%%%%%%%%%%%%%%%%%%%%%%%%%%%%%%%%%%%%%%%%%%%%%%%%%%%%%%%%%%%%%%%%%%%%%%%
\end{document}
